% !TEX root = ../main.tex

Neural networks are widely used for machine learning due to the large amount of data available, especially in deep learning. However, when it comes to solving non-continuous control tasks in the field of reinforcement learning, neural networks can be counterintuitive since they are continuous. To address this issue, this project explores the hypothesis that binary trees, as discontinuous models, could have an advantage in solving these tasks. To further develop this model, the project introduces an insertion strategy that allows the tree to grow dynamically when tasks become too complex for its current structure. This technique should be a first step toward architecture search for binary trees. The approach has been tested on various benchmarks and the results are promising, indicating that a growing binary tree could be an efficient model for solving many control tasks. The project provides valuable insights into the use of the binary tree as an alternative model for reinforcement learning and highlights the potential benefits of using dynamic structures such as growing the trees for efficient and effective learning.