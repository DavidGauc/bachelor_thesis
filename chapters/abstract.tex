% !TEX root = ../main.tex

Neural networks have become a powerful tool for machine learning, particularly with the large amount of data now available. Deep learning has proven to be an effective method for many applications in various domains, but neural networks have also been used in the field of reinforcement learning to solve non-continuous control tasks. This approach is counterintuitive, since neural networks are continuous.The hypothesis is that discontinuous models such as binary trees would have an advantage in addressing these tasks. To test this hypothesis, a new function has been introduced that allows the tree to grow dynamically when tasks are too complex to be solved by its current structure. This is an important step towards architecture search for binary trees, and has been tested on different benchmarks with promising results. The results indicate that a growing binary tree could be an efficient model for solving many control tasks.
Overall, the study provides insights into the use of discontinuous models for reinforcement learning, and the potential benefits of using dynamic structures such as growing binary trees for efficient and effective learning.